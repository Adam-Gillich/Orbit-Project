\documentclass[12pt,a4paper]{article}

\usepackage[utf8]{inputenc}
\usepackage[magyar]{babel}
\usepackage[T1]{fontenc}
\usepackage{amsmath, amssymb, physics, verbatim, graphicx}
\usepackage{float}

\usepackage[backend=biber, style=phys]{biblatex}
\addbibresource{/home/adam/work/LaTeX/Mecha/Orbit project/Orbit_project_conclusion/References.bib}
\usepackage{url}
\usepackage{xcolor}
\usepackage{listings}

\usepackage[a4paper, top=2.5cm, bottom=2.5cm, left=2.5cm, right=2.5cm]{geometry}
\usepackage{titling}
\usepackage{ragged2e}
\usepackage{newtxtext}
\usepackage{newtxmath}
\justifying
%\onehalfspacing
\setlength{\droptitle}{-3cm}


\title{Orbitális pályára állás}
\author{Gillich Ádám}
\date{\today}

%-------------------------------------------------------------------------------------------

\begin{document}
\nocite{eaglepubs,Teofilatto,nasaSpecificImpulse,nasaRocketThrust,OrbitalMechanicsByWeber,Stanford,Hohmann,eccenticityWiki,pegas,rssTwr,eccenticityForum,dragGithub,mikeVisViva,OrbitProject,brentq,RD-253,RD-0124,KSP_deltaV_map,RSS_deltaV_map}

\maketitle
%-------------------------------------------------------------------------------------------
\section*{Bevezetés}
Munkám célja az alacsony Föld körüli és Kerbin körüli orbitális pályára (LEO és LKO) álláshoz szükséges 
$\Delta v$ kiszámítása. A projekt lényegesen változott az eredeti célhoz képest. A programozott röppálya 
el lett vetve, mert túl új és komplikált, valamint amint kiderült \citeauthor{Teofilatto} munkája kiváló, 
de nem pontosan erre a célra tervezték. Így a kísérleteknél kézi vagy modok segítségével fél automata 
irányítást alkalmaztam, az alapján, hogy mi adta a jobb eredményt.\medskip\\
A dokumentum felépítése:
\begin{enumerate}
    \item Elmagyarázom az alapfogalmakat, amik kellenek a megértéshez.
    \item Egyszerűen elmagyarázom, hogy hogyan állítanak pályára egy rakétát, amit összevetek azzal, amit a 
    számításaimban használtam.
    \item Elmagyarázom a Python kódom lényegi részét.
    \item Bemutatom a kísérlethez használt rakétákat, és a kísérletet. Bemutatom az eredményeket.
    \item Ezek után kiértékelem, összevetem az eredményeket.
\end{enumerate}
A projektben használt és keletkezett anyag minden formáját tőlem telhető legrészletesebben igyekszem 
feltölteni a \cite{OrbitProject} forrás alatt található GitHub repository-ba.\footnote[1]{A projekt "kódneve" 
\textit{'Orbit Project'} volt, így ezen a néven is előfordulhat}.

%-------------------------------------------------------------------------------------------
\section{Fogalomtár}

\begin{itemize}
    \item \textbf{Orbitális pálya (pálya)} — olyan általánosan ellipszis alakú pálya ami egy égitest körül 
    helyezkedik el, nem érintve annak felszínét. Kialakításához elsősorban sebességre van szükség, 
    magasság csak a légkör miatt követelmény. Lehet rá folyamatos szabadesésként is gondolni,
    ami azért lesz folyamatos, mert olyan sebességgel haladunk a bolygó felszíne felett, hogy kifordul
    alattunk a bolygó. Így sosem esünk vissza.
    \item $\mathbf{\Delta v}$ — Egyszerűen úgy lehet rá tekinteni, mint a rakéta üzemanyaga. Hasonló módon, mintha 
    egy autóban ülve nem a megmaradt liternyi üzemanyagot néznéd meg, hanem azt, hogy hány $km$ utat tud
    még megtenni. És más autóba átülve nem kell gondolkodni, hogy az autó tulajdonságait tekintve mennyire
    elég $5\, liter$ benzin, ha tudod, hogy még $50\, km$-t meg tudsz tenni vele. A Tziolkowski egyenletből
    következik, hogy a rakéta az űrben a sebességét tudja megváltoztatni, és arra a következtetésre lehet
    jutni, hogy érdemes egy rakéta "üzemanyagát" $m/s$-ban mérni, és a $\Delta v$ pedig az a sebességváltozás,
    amit a rakéta ideális esetben vákuumban el tud érni. Általában nem is kell foglalkozni azzal,
    mi van, ha nem ideális esetben vagyunk, de itt pont ez lesz a helyzet.
    \item \textbf{Gravitációs Forduló} — A pályára állás az a része amikor hagyjuk, hogy a gravitáció fordítsa a rakétát.
    Azért fordítja, mert ha bedöntjük a rakétát, akkor a gravitáció a sebességvektorral nem lesz párhuzamos, 
    így azon kívül, hogy annak ellentart, még forgatja is azt.
    \item \textbf{Hohmann módszer} — Egy nagyon hatékony módszer a $\Delta v$ kiszámítására, amikor egyik 
    körpályáról egy másikra megyünk. De ennek két fele van, ugyanis az űrben nem tudjuk egy ponton megemelni a 
    pályánkat, így megemeljük egy ponton, majd fölmegyünk az \textit{Apoapsis}-ra (lásd lejjebb) és ott
    is gyorsítunk egyet, így lesz újra körpályánk. Ebből nekünk az utóbbi eset kell majd.
    \item \textbf{Apoapsis} — a pálya legmagasabb pontja (általánosan)
    \item \textbf{Periapsis} — a pálya legalacsonyabb pontja (általánosan)
    \item \textbf{Prograde} — haladási irány
    \item \textbf{liftoff} — a rakéta elengedésének pillanata
    \item \textbf{TWR} — Tolóerő–Tömeg arány, a \textit{Thrust-to-Weight Ratio} rövidítése
    \item $\mathbf{I_{sp}}$ — A specifikus impulzus, tolóerő osztva a súllyal ($I/mg$) \cite{nasaSpecificImpulse},
    ez egy hajtóműre jellemző mennyiség, minden hajtóműnek van sajátja. A legegyszerűbben úgy lehet rá 
    gondolni mint a hajtómű teljesítménye, hatékonysága. A nagyobb $I_{sp}$ jobb.
    \item $\mathbf{R}$ — a bolygó sugara
    \item $\mathbf{\mu}$ — $GM$, ahol $G$ a gravitációs állandó, $M$ pedig a bolygó tömege, neve 
    \textit{standard gravitációs paraméter}, minden égitestnek van sajátja. Ezt a jelölést bevezetve kicsit 
    rövidebbé, és egyszerűbben érthetővé válnak a formulák.
    \item \textbf{Analitikus megoldás} — Egy olyan megoldás, ami pontosan adja vissza az értéket
    \item \textbf{Numerikus megoldás} — Ha a problémának nincs analitikus megoldása, vagy nehéz megtalálni az analitikus megoldást,
    a közelítő számítások mellett (például: Taylor-sorfejtés) numerikus módszerrel is célt lehet érni. 
    A numerikus módszerek nem adnak pontos eredményt, de ismerve a mérések pontatlanságát és a rendszert befolyásoló faktorokat,
    az eredmény kellően pontos lehet. A legtöbb módszernél meg lehet adni egy $\delta$ vagy $\varepsilon$ paramétert,
    hogy mennyire közelítse meg a valós megoldást.
    \item \textbf{KSP, és RSS} — ez talán a legfontosabb. A kísérleteket a Kerbal Space Program alap és modolt
    változatában végeztem el, amikor azt írom KSP akkor az alap játékra gondolok, és amikor azt írom RSS akkor
    a modoltra. RSS onnan jön hogy az egyik fő mod a \textit{Real Solar System}.
    
\end{itemize}

%-------------------------------------------------------------------------------------------
\section{Elméleti háttér}

\subsection{Hogyan működik egy rakéta fellövése?}
Ha körpályára akarunk állni a Föld körül, akkor 3+1 lépésből áll.
\begin{enumerate}
    \item Függőleges emelkedés: A rakéta \textit{liftoff} után nem csinál semmit csak emelkedik egy bizonyos
    pontig.\footnotemark[2]
    \item Gravitációs forduló: A rakéta egy bizonyos, jól optimalizált szöggel bedől, és azután követi a 
    \textit{Prograde} irányát. Hagyja, hogy a gravitáció segítsen a fordulásban, amíg a légkör hatása 
    elhanyagolhatóvá nem válik.
    \item Irányított szakasz: Ha a légkör már elhanyagolható, a rakétát átadják egy algoritmusnak, ami 
    valós időben korrigálja a pályát, és nagy valószínűleg hamar áttér közel vízszintes repüléshez, azért 
    hogy sebességet gyűjtsön az orbitális pálya kialakításához.\\ A \cite{pegas} forrásban \citeauthor{pegas}
    az űrsikló dokumentációi alapján készített egy ilyen algoritmust pályára álláshoz.\footnotemark[3]
    \item Siklás\footnotemark[4] és körpálya: Ha megtette az algoritmus amit tudott, akkor vár, ez a siklási
    szakasz. Addig vár amíg el nem éri a rakéta az \textit{Apoapsis}-t, ott újra begyújtja a hajtóművét és 
    kialakítja a körpályát.
\end{enumerate}
\footnotetext[2]{Meghatározható sebességben, időben, magasságban, és talán másban is de ezek a gyakori 
kézenfekvő opciók.}
\footnotetext[3]{Amit nekem sajnos nem sikerült működésbe hozzak.}
\footnotetext[4]{\textit{Coasting Flight}}

%-------------------------------------
\subsection{A számítások fizikai háttere}
\subsubsection*{Változások}
Eredetileg a \cite{eaglepubs} forrás 57. fejezetében található módszert terveztem használni a $\Delta v_{total}$ 
kiszámítására, ami még mindig egy lehetőség, de én végül eltértem tőle. Majd később részletezem a 
pontos módszert.\\
Az \cite{eaglepubs} forrásban szereplő módszer:
\begin{equation}
    \Delta v_{total} = \Delta v_K + \Delta v_U + \Delta v_g + \Delta v_d + \Delta v_{steer} - \Delta v_{lat}
\end{equation}

\bigskip
Az általam írt Python kód és a hozzá tartozó fizika erősen összekapcsolódik, de megpróbálom a lehető 
legjobban szétválasztani őket.

%-------------------------------------
\subsubsection*{\citeauthor{Teofilatto} munkájának célja}
\citeauthor{Teofilatto} munkájának a célja egy analitikus megoldás volt a \textit{Gravitációs Fordulóra}, 
ehhez néhány egyszerűsítést kénytelenek voltak bevezetni. Például, hogy a légellenállás elhanyagolódik, 
de később látjuk majd, hogy nem is olyan nagy mértékű. Valamint, hogy a tolóerő-tömeg arány (\textit{TWR}) 
egy átlagos értékkel leírható. Ami azért nem pontos, mert a hajtómű tolóereje nem konstans, és a tömeg sem. 
A \cite{nasaRocketThrust} forrás alapján a tolóerő, ahogy a nyomás csökken, úgy növekszik, míg a tömeg 
folyamatosan fogy. Így a \textit{TWR} folyamatosan nő. Ezt bizonyos körülmények közt akár egy konstans 
értéken lehetne tartani amennyiben a hajtómű tolóereje manuálisan is állítható.

%-------------------------------------
\subsubsection*{A rakéta összes $\Delta v$-jének kiszámítása}
A két fokozat összesen kifejthető $\Delta v$-je kiszámítható az \cite{eaglepubs} és \cite{Teofilatto} forrásban is
 található Tziolkowski formula egy változatával. A két fokozatnak külön kell kiszámítani a $\Delta v$-jét, majd 
összeadni őket ahhoz, hogy megtudjuk, mennyi $\Delta v$ kifejtésére képes a rakéta.
\begin{equation}
    \Delta V_{\mathrm{Tz}} = g_0\, I_{\mathrm{sp}\,k}\, \ln\!\left(\frac{M_k}{M_{kf}}\right)
\end{equation}
Ahol a $g_0$ a gravitációs gyorsulás föld közelben, $I_{\mathrm{sp}\,k}$ a hajtómű specifikus impulzusa, $M_k$ a 
fokozat teljes tömege\footnotemark[5], $M_{kf}$ pedig vég tömeg.\footnotemark[5]
\bigskip

\footnotetext[5]{$M_k$: üzemanyag, struktúra, hasznos teher; $M_{kf}$: struktúra,
 hasznos teher (száraztehernek is nevezik)}

%-------------------------------------
\subsubsection*{A mérés logikája}
Maga a cél a Föld körül $250\, km$, és $120\, km$ magasságú körpályát kialakítani a lehető legkevesebb elhasznált $\Delta v$-vel.
Ezt elnevezem $\Delta v_{total}$-nak.\\
\smallskip

A rakéta képzeletbeli útja:
\begin{itemize}
    \item Emelkedünk
    \item Gravitációs forduló, \citeauthor{Teofilatto} analitikus módszere alapján
    \item Az \textit{Apoapsis} eléri a 250 km-t
    \item A kialakult \textit{Apoapsis} és \textit{Periapsis} segítségével meghatározzuk, hogy Hohmann 
    módszerrel mennyi $\Delta v$ körpályává tenni a jelenlegi ellipszisünket.
    \item Összeadjuk a fokozatok által elhasznált $\Delta v$-t, meghatározva a $\Delta v_{total}$-t
\end{itemize}
A feltevés az, hogy lesz egy ideális pálya $\gamma_f$ vég-röppályaszöggel, $v(z_{2f})$ sebességgel, 
és $R_{F\ddot{o}ld}+h(z_{2f})$ sugárral aminek az \textit{Apoapsis}-a éppen $250\, km$ lesz. És az ezt követő 
Hohmann módszer után a $\Delta v_{total}$ a minimum lesz.\\

\begin{figure}[H]
    \centering
    \includegraphics[width=1 \textwidth]{Pictues/3_stages (1).png}
    \caption{A röppálya elméleti ábrája}
    \label{fig:trajecroty}
\end{figure}


\citeauthor{Teofilatto} módszere erre nem ad választ, viszont ki tudom számolni, hogy a hajtóművek égetése után 
végül mekkora sebességgel, $v(z_{2f})$ megyek milyen magasságban, $h(z_{2f})$, a végső szög $z_{2f}$ függvényében,
ami a $\gamma_f$, a vég-röppályaszög, egy ügyesen átalakított formája.\footnotemark[6]
\smallskip

\footnotetext[6]{Az indíték ilyen módú jelölésére és az átalakítás módszerére a \cite{Teofilatto} forrás 
7. oldalán található válasz.}

A sebesség a mozgás végén:
\begin{equation}
    v(z_{2f}) = A_2\, z_{2f}^{\bar{n}- 1}\, (1 + z_{2f}^2)
\end{equation}

A magasság a mozgás végén:
\begin{equation}
h(z_{2f}) = h(z_{20}) + \frac{A_2^2}{g}   \left[ 
                                    \frac{z^{2\bar n-2}}{2\bar n-2} - \frac{z^{2\bar n+2}}{2\bar n+2} 
                                \right]_{z_{20}}^{z_{2f}}
\end{equation}
Ahol $\bar{n}$ az átlagos \textit{TWR}, és $A_2$ nincs megnevezve de abból ahogy használva van úgy 
lehetne elnevezni, mint egy röppálya skálázó paraméter, ami mindig egy a fokozatra jellemző.
\medskip

\citeauthor{Teofilatto} azzal számoltak, hogy $\gamma_f$ a mozgás végén ismert. De mivel a mi helyzetünkben 
nem ismert, ezért a problémát numerikusan kell megoldani.

A módszerükkel kiszámítható a $\Delta v_g$, tehát az a $\Delta v$ amit a gravitáció miatt 
elvesztünk.
\begin{equation}
    \Delta v_g = \Delta v_{Tz1} + \Delta v_{Tz2} - v(z_{2f})
\end{equation}
Tehát a két fokozat által összesen elhasznált $\Delta v$ mínusz a ténylegesen elért sebesség.
\medskip

De még kellenek a pálya adatok.\\
A két fontos adat a Hohmann módszer szempontjából az \textit{Apoapsis} és a \textit{Periapsis}\\
Ha megvan az excentricitás akkor a \cite{OrbitalMechanicsByWeber} forrás alapján kettőt úgy számoljuk ki, hogy:
\begin{equation}
    r_{Ap} = a(1 + e); \qquad r_{Pe} = a(1 - e)
\end{equation}
Ahol $a$ a félnagytengely és $e$ az excentricitás.
\smallskip

Szintén a \cite{OrbitalMechanicsByWeber} forrásból a félnagytengely a következő módon felírható:
\begin{equation}
    a = - \frac{\mu}{2 E}
\end{equation}
Ahol $E$ a mechanikai energia.

%-------------------------------------
\subsubsection*{Excentricitás vektor}
A \cite{eccenticityWiki} forrásból megállapítható, hogy az excentricitás vektor a következő:
\begin{equation}
    \mathbf{e} = \frac{\mathbf{v} \times \mathbf{L}}{\mu} - \frac{\mathbf{r}}{r}, \qquad
    \mathbf{L} = \mathbf{r} \times \mathbf{v}
\end{equation}
Négyzetezzük:
\begin{align*}
    e^2 = \left( \frac{\mathbf{v} \times \mathbf{L}}{\mu} - \frac{\mathbf{r}}{r} \right)^2, \qquad (a - b)^2\\
        = \frac{(\mathbf{v} \times \mathbf{L})^2}{\mu^2} - 2 \cdot \frac{1}{\mu} \frac{\mathbf{r}}{r} \cdot (\mathbf{v} \times \mathbf{L}) + 1, 
\end{align*}
\[
    \mathbf{v} \bot \mathbf{L},\qquad \mathbf{r} \cdot (\mathbf{v} \times \mathbf{L}) = 
    \mathbf{r} \cdot (\mathbf{v} \times (\mathbf{r} \times \mathbf{v})) = 
    (\mathbf{r} \times \mathbf{v}) \cdot (\mathbf{r} \times \mathbf{v}) =
    (\mathbf{r} \times \mathbf{v})^2 = L^2
\]
\begin{equation*}
    e^2 = \frac{v^2 L^2}{\mu^2} - \frac{2 L^2}{\mu r} + 1 
        = 1 + \frac{L^2}{\mu^2} \left( v^2 - \frac{2\mu}{r} \right)
        = 1 + \frac{L^2}{\mu^2}\, 2 E
\end{equation*}
Bevezetve az $E = \frac{v^2}{2} - \frac{\mu}{r}$-t az excentricitás vektor nagysága a következő:
\begin{equation}
    e = \sqrt{1 + \frac{2 E\, L^2}{\mu^2}} 
\end{equation}
Ahol $E$ a mechanikai energia:
\begin{equation*}
    E = \frac{v^2}{2} - \frac{\mu}{r}
\end{equation*}
És $L$ pedig a perdület:
\begin{equation*}
    L = \mathbf{r} \times \mathbf{v} = r\, v \cos\gamma
\end{equation*}
\medskip

\noindent
Ez a mi esetünkre átalakítva:
\begin{equation}
    E = \frac{v(z_{2f})^2}{2} - \frac{\mu}{R + h(z_{2f})}, \qquad L = \big[R + h(z_{2f})\big]\ v(z_{2f})\, \cos\gamma_f
\end{equation}

Miután megvannak a pályaadataink még mindig ellipszis pályán keringünk, amit körpályává kell alakítani.
Ehhez a Hohmann módszert alkalmazzuk, a \cite{Hohmann} forrás alapján:
\begin{equation}
    \Delta v_H = \sqrt{\frac{\mu}{r_{Pe}}} \left( \sqrt{\frac{2 r_{Ap}}{r_{Ap} + r_{Pe}}} -1 \right)
\end{equation}

\medskip

Nagyon jó mostmár meg tudjuk állapítani a pályaadatainkat a mozgás végi adatokból, és körpályára tudunk állni.\\
Tehát mégegyszer: A feltevés az, hogy lesz egy ideális pálya $\gamma_f$ vég-röppályaszög, 
$v(z_{2f})$ sebességgel, és $R+h(z_{2f})$ sugárral aminek az \textit{Apoapsis}-a éppen 250 km lesz. 
És az ezt követő Hohmann módszer után a $\Delta v_{total}$ a minimum lesz.
\medskip

Ehhez numerikusan optimalizálni kell öt paramétert:\\
\indent $\Delta v_{total}$, $r_{Ap}$, $\gamma_f$, $z_{20}$, $M\%$-et.\\
\indent $\Delta v_{total}$, $r_{Ap}$, $\gamma_f$-val már találkoztunk, $z_{20}$ a röppályaszög az első 
fokozat leválasztásánál $M\%$ pedig abból következik, hogy nem minden üzemanyagunkat akarjuk elégetni 
csak azt amennyi kell. A név abból származik hogy ez tényleg egy százalék tehát például 56.66\%, ebben 
az esetben a második fokozat 56.66\%-át használtuk fel.

%---------------------------------------------------------------
\subsection{Numerikus számítási folyamat}
Az számítási folyamatot szinte teljesen én készítettem, apukám, Gillich Péter segédkezett a numerikus 
optimalizációban\footnotemark[7].\\

A mérés módja az (1) egyenletben leítrakból úgy változott, hogy most azt mérem mennyi $\Delta v$ lett elhasználva, 
nem pedig tételesen kiszámolom mennyi lenne a $\Delta v_{total}$.

%-------------------------------------
\subsubsection*{Kis kiegészítés}
Azzal, hogy tudjuk állítani a $M\%$-et, két dolog fog változni \textbf{1) $M_{2f}$},\, \textbf{2) $t_{b2}$}, tehát
a második fokozat vég tömege és az idő ameddig égetjük az üzemanyagot.\\
$M_{2f}$ kilogikázható:
\begin{equation*}
    M_{\ddot{u}} = M_{20} - M_{2\, \mathrm{dry}}
\end{equation*}

\begin{equation*}
    M_{\ddot{u}\, maradt} = M_{\ddot{u}} [1 - (M\% / 100)]
\end{equation*}

\begin{equation}
    M_{2f} = M_{2\, \mathrm{dry}} + M_{\ddot{u}\, maradt}
\end{equation}
Ahol $M_{\ddot{u}}$ az üzemanyag tömege, $M_{2\, \mathrm{dry}}$ a száraz tömeg, $M_{2f}$ pedig a vég tömeg.\\

\medskip
\noindent
A másik a $t_{b2}$ azaz a második fokozat kiégésének ideje. A \cite{nasaSpecificImpulse} és \cite{nasaRocketThrust}
forrásból meg lehet határozni $\dot{m}$-t, ebből következik:
\begin{equation}
    t_{b2} = \frac{M_{\ddot{u}\, haszn\acute{a}lt }}{\dot{m}} = 
    \frac{M_{20} - M_{2f}}{F_{\text{Tol\'oer\H{o}}} / (g_0\, I_{sp})} = \frac{(M_{20} - M_{2f})\, g_0\, I_{sp} }{F_{\text{Tol\'oer\H{o}}}}
\end{equation}

\footnotetext[7]{Például a \texttt{find\_max\_not\_NaN} függvény 
egy pair codinggal készített, erre az esetre szabott rekurzív függvény.}

%-------------------------------------
\subsubsection*{Numerikus módszerek összefoglalója}
Az öt paraméter, amit optimalizálni kell: $\Delta v_{total}$, $r_{Ap}$, $\gamma_f$, $z_{20}$, $M\%$\\
A számítás menete, sorrendben:
\begin{enumerate}
    \item Meg kell találni $M\%$-nek azt az intervallumát, ahol még van olyan $\gamma_f$, amivel elérjük a kívánt $r$-t.
    \item Választunk egy $M\%_i$-t és egy $\gamma_i$-t
    \item Kiszámoljuk a $\gamma_i$-hez tartozó $z_{20}$, közép röppályaszöget
    \item Megnézzük, hogy ezzel a $\gamma_i$-vel eléri-e az $r_{Ap}$ a kívánt $r$-t // Ha nem, akkor másik $\gamma_i$
    \item Ha igen, akkor megnézzük, mennyi a $\Delta v_{total}$ // Kevesebb vagy több
    \item Választunk egy másik $M\%_i$-t
\end{enumerate}
Az optimalizáció végén megkapjuk a minimális $\Delta v$-t, amit el kell használnunk a pályára álláshoz.\\

\begin{figure}[H]
    \hspace{-2.5cm}\includegraphics[width=1.3 \textwidth]{Pictues/numerical-calc.png}
    \caption{A számításokat végző program ábrája}
    \label{fig:num}
\end{figure}

\subsubsection*{Alkalmazott numerikus módszer típusok}
\textbf{Brent}: A paraméterek közül kettő ($z_{20}$ és a $r_{Ap}$) a scipy.optimize/\textit{brentq} 
függvényével lett kiszámítva.\\
A \textit{brentq} függvény nullára optimalizáláskor lehet használni, ami összetett, de hatékony. Egy biztonságos 
felező módszert ötvöz inverz kvadratikus extrapolációval.\footnotemark[8]\\
\indent Ezek közül az $r_{Ap}$ a legkézenfekvőbb, hiszen van egy $r$ amit el akarunk érni, így:
\begin{equation}
    f_{Ap} = r - r_{Ap}
\end{equation}
Ha ez a különbség 0, akkor megvan az a $\gamma_i$, amivel el lehet érni a kívánt $r$-t.\\
\smallskip

\footnotetext[8]{A módszer Richard Brent által lett kidolgozva \cite{brentq}}

A másik a $z_{20}$, amit \citeauthor{Teofilatto} számításaiból származik.\\
A lényeg, hogy létezik olyan $f_{1, 2}$ függvény, ami:
\begin{align*}
    t_{b1} = f_1(z_{20})\\
    t_{b2} = f_2(z_{20})\\
    \frac{t_{b1}}{t_{b2}} = \frac{f_1}{f_2}
\end{align*}
Tehát:
\begin{equation}
    f(z_{20}) = \frac{t_{b1}}{t_{b2}} - \frac{f_1}{f_2}
\end{equation}
Ha ez a különbség 0, akkor megvan a $\gamma_i$-hez tartozó $z_{20}$ szög.\\
\smallskip

\textbf{minimize\_scalar}: Maga a $\Delta v_{total}$ nem számítható ki \textit{brentq}-val, minimumot kell keresni, nem 
zéruspontot. Ezért a scipy.optimize egy másik függvényét hívtuk segítségül: \textit{minimize\_scalar}.
\smallskip

\textbf{find\_max\_not\_NaN}: Ez a függvény találja meg a $M\%$-nek azt az intervallumát, ahol a 
$f_{Ap} = 0$ értelmezett. Ez egy külön erre a célra készített függvény, ami a felezőmódszerhez hasonlóan működik,
de nem zéruspontot keres, hanem az értelmezési tartomány határát.
\smallskip

\textbf{A megoldások}: $\gamma_f$ és $M\%$ nincs használva ilyen numerikus módszerben, csupán az intervallumukat
keressük meg numerikusan. Tehát ezek a keresett változók és ha megvan a $\gamma_f$ és $M\%$, akkor megkapom a 
megoldást, ami a $\Delta v_{total}$.


%-------------------------------------
\subsubsection*{A számítás általánossága}
Természetesen valamennyire általános ez a megoldás. A két fokozatú rakétákra, amiknek legalább az első fokozat
egészére szükségük van a pályára álláshoz. \citeauthor{Teofilatto} módszere kiterjeszthető N fokozatra is, és
a kellő tudással meg lehet azt csinálni, hogy több fokozatból engedjünk állítani felhasznált százalékot.
De egy általánosítása volt a számításnak, hogy az első fokozatra szükség van. Ennek az indoklása annyi, hogy 
sem KSP-ben, sem RSS-ben nem volt olyan első fokozat, ami elérni az első kozmikus sebességet a bolygó körül, 
de lehetne. Vagy éppen három fokozata van, valóságban vannak rá példák.\footnotemark[9]\\
Illetve nem kizárt hogy ki lehet terjeszteni magas Föld kürüi pályákra is a számítást, de ez már nem tartozik a 
projekt kereteibe.

\footnotetext[9]{Az indíték az, hogy a hajtóművek $I_{sp}$-je általában 280-350 köré esik. És körülbelül 
$I_{sp} \cdot 10$\, érdemes egy fokozat $\Delta v$-jét tenni, és ehhez Föld körül három fokozatra van szükség. 
Ilyen a Proton rakéta család, és a méltán híres Saturn V, és a Soyuz rakéták is.}

%-------------------------------------
\subsubsection*{Talált hibák a \cite{Teofilatto} forrásban}
\citeauthor{Teofilatto} munkája lenyűgöző, de nem hibátlan. Ami megnehezítette a munkámat és akár több nappal is 
hátravetett.\\
Mivel az én példáim is két fokozatra épülnek, gyorsan le is tekertem a 4. fejezetig (\textit{Analytic Derivation
 of Launcher Trajectories}). Ahol pont két fokozatos példát ecsetelgetnek. Itt a (20) egyenlet teljesen el van írva.
 $\bar{n}_{1,2}$ (átlagos \textit{TWR}) össze-vissza van írva, és hiányoznak a zárójelek. 
\begin{equation}
t_{b2} = \frac{A_2}{g} \left( \frac{z_{2f}^{\bar{n}_2-1}}{\bar{n}_2-1} + \frac{z_{2f}^{\bar{n}_2+1}}{\textcolor{red}{\bar{n}_1}+1} 
- \frac{z_{20}^{\bar{n}_2-1}}{\bar{n}_2-1} \textcolor{red}{+} \frac{z_{20}^{\bar{n}_2+1}}{\textcolor{red}{\bar{n}_1}+1} \right) 
\end{equation}
Ez a ChatGPT-nek
rögtön feltűnt, mikor rákérdeztem, hiszen mint arra felhívta a figyelmemet, ez egy Newton-Leibniz tételes 
kiértékelés eredménye. Mint az megfigyelhető a (26) egyenletben is, ahol 3 fokozatra írnak példát. \\
És itt szerencsére a (25) egyenletben javítják, amit elírtak.
 \begin{equation}
t_{b2} = \frac{A_2}{g} \left[ \left( \frac{z_{2f}^{\bar{n}_2-1}}{\bar{n}_2-1} + \frac{z_{2f}^{\bar{n}_2+1}}{\bar{n}_2+1} \right)
- \left( \frac{z_{20}^{\bar{n}_2-1}}{\bar{n}_2-1} + \frac{z_{20}^{\bar{n}_2+1}}{\bar{n}_2+1} \right)  \right]
\end{equation}
Ahol a $t_{b2}$ a második fokozat kiégésének ideje; $A_2$ nincs megnevezve, de abból, ahogy használva van, úgy 
lehetne elnevetni, mint egy röppálya skálázó paraméter, ami mindig egy a fokozatra jellemző; $z$ pedig a 
röppályaszög, $z_{2f}$ a második fokozat kifogyásánál, $z_{20}$ pedig az első fokozat leválasztásánál.\\
Egy másik hibát szintén ugyanezen a helyen én vettem észre: $f_{1,2}$ meg volt cserélve:
\begin{equation}
    \frac{t_{b2}}{t_{b1}} = \textcolor{red}{\frac{f_1}{f_2}}
\end{equation}

Helyesen az egyenlet:
\begin{equation}
    \frac{t_{b2}}{t_{b1}} = \frac{f_2}{f_1}
\end{equation}
Egy harmadikat megint a ChatGPT. A $h(z)$ a (14) egyenlettől végig rosszul van írva. Ha megfigyeljük a 8. 
oldal legalját, akkor ott van a $\partial_z(h)$, amit integrálni kell $z$ szerint, hogy megkapjuk $h(z)$-t.
Az integrálást elvégezve:
\begin{equation}
    \int \frac{dh}{dz} dz =\frac{A^2}{g} \int \left( z^{2\bar n-3} - z^{2\bar n+1} \right) dz
\end{equation}

\begin{equation}
h(z) = h(z_0) + \frac{A^2}{g}   \left[ 
                                    \frac{z^{2\bar n-2}}{2\bar n-2} - \frac{z^{2\bar n+2}}{2\bar n+2} 
                                \right]_{z_0}^{z_f} 
\end{equation}
Náluk a pirosra színezett blokkban tűnt el a kettes:
\begin{equation}
h(z) = h(z_0) + \frac{A^2}{g}   \left[ 
                                    \frac{z^{2\bar n-2}}{2\bar n-2} - \textcolor{red}{\frac{z^{2\bar n+2}}{\bar n+2}} 
                                \right]_{z_0}^{z_f} 
\end{equation}

%-------------------------------------------------------------------------------------------
\section{A kísérletek}
A kísérletek elvégzéséhez a Kerbal Space Program nevű "játékos szimulátort" használtam.
Egy alap játékos konfigurációban és egy modokkal megváltoztatott konfigurációban, ami lehetővé tette, hogy a 
Földön is teszteljem azt, amit kiszámítottam, illetve mindezt igazi hajtóművekkel tegyem.

%-----------------------------------------------------
\subsection{A rakéták}
A rakéták pontos számítás szempontjából lényeges adata megtalálható egy YAML fájlban \cite{OrbitProject}
GitHub repository-ban a Python kód mellett.
%-------------------------------------
\subsubsection*{KSP}
Az alap játékban egy a Gemini programban használtra hajazó rakétát raktam össze.\\
A fokozatok hajtóművei:
\begin{enumerate}
    \item "Bobcat"
    \item "Cheetah"
\end{enumerate}

\begin{figure}[H]
    \centering
    \includegraphics[width=0.5 \textwidth]{Pictues/kemini.png}
    \caption{Képek a rakétáról a hangárban.}
\end{figure}

KSP-ben a kilövőállás hajszálra rajta van az egyenlítőn, így nem kell egyszerűsíteni sem a $\Delta v_{lat}$-ön.
%-------------------------------------
\subsubsection*{RSS}
Itt nem céloztam semmit, amire hasonlítania kéne a rakétának. A hajtóművek orosz gyártásúak, de csak 
azért, mert azok tűntek a legmegfelelőbbnek.\\
A fokozatok hajtóművei:
\begin{enumerate}
    \item RD-253
    \item RD-0124
\end{enumerate}

\begin{figure}[H]
    \centering
    \includegraphics[width=0.5 \textwidth]{Pictues/RSSorbitProject.png}
    \caption{Képek a rakétáról a hangárban.}
\end{figure}

Az RD-253 hajtóművek a Proton rakéta első fokozatát alkották, egy nagyon megbízható, sokat használt
hajtómű. A "Salyut", "Mir" programokban is alkalmazták. 2012-ben lett kivezetve, azóta a fejlesztett
RD-275M van szolgálatban. \cite{RD-253}
\smallskip

Az RD-0124 a Soyuz rakéták harmadik fokozatát alkotja, és még a mai napig használatban van. És én is 
ugyanarra a célra használom, ami az hogy egy felsőbb fokozatos hajtómű legyen. A hajtómű egy érdekes
konstrukció, ugyanis egy turbópumpa van, ami négy fúvóka égésterébe pumpálja az üzemanyagot. Tehát a 
négy fúvóka ellenére ez mégis egy hajtómű. \cite{RD-0124}
\medskip

\smallskip
Fontos szempont volt a hajtómű választásnál, hogy a második fokozat minimum kétszer gyújtható legyen.
Mert az RD-253, mivel első fokozatra tervezett, ezért nem számolnak azzal, hogy le lesz állítva, míg az 
RD-0124 többször (kétszer) begyújtható a körpálya kialakítása miatt.

RSS-ben a legtöbb valóságban használt kilövőállás a rendelkezésemre áll, így a Francia Guyanában 
található Kourou kilövőállást használtam. De még az elején Cape Canaveral-nél is szórakoztam.
\cite{OrbitProject}

\subsection{Az irányítás}
Mint azt a legelején említettem vagy kézi vagy fél automata irányítást fogok használni.\\
Kézi KSP-ben, de csak mert az hozta a legkisebb $\Delta v_{total}$-t.\\
És RSS-ben a \textit{MechJeb2} nevű mod segítségével egy fél automata vezérlést.\\

Az eredményeket szintén a \textit{MechJeb2} rögzítette.

%-------------------------------------------------------------------------------------------
\section{Kiértékelés}

A kiértékelést a számításokkal kezdem, és azután következik a kísérlet, majd az összehasonlítás,
tanulságok levonása.

\subsection{Számítások eredménye}

Sok adat volt végül kiírva, segítve a hibák megtalálását ezt megtartottam a végső verzióban
is de itt csak a legfontosabbakat fogom listázni.\footnotemark[10]

\begin{table}[h]
    \centering
    \begin{tabular}{|c|c|c|c|}
        \hline
        \textbf{Paraméter} & \textbf{KSP} & \textbf{RSS} & \textbf{M.egység} \\
        \hline
        $\Delta v_{total}$ & 3223.72 & 8930.49 & m/s \\
        \hline
        $\Delta v_g$ & 651.5 & 1102.75 & m/s \\
        \hline
        $\Delta v_H$ & 282.05 & 42.2 & m/s \\
        \hline
        $\gamma_f$ & 7.16 & 0.31 & fok \\
        \hline
        $M\%$ & 64.251 & 85.234 & \% \\
        \hline
    \end{tabular}
    \caption{Fontosabb mért adatok eredménye.}
    \label{tab:numres}
\end{table}

A számításokban azt figyeltük meg, hogy $M\%$ szélső értékhez közelít, ezért volt szükség a \textit{find\_max\_not\_NaN}
függvényre. Mert egy bizonyos százalék felett nincs olyan $\gamma_f$ amivel az \textit{Apoapsisunk} elérné a $250\, km$-t. Ekkor a 
\textit{brentq} hibát ad, miszerint az intervallum nem halad át a nullán.\\
Ez azt jelenti, hogy a körpálya kialakítására szükséges $\Delta v$-t minimalizálni kell. Tehát az első gyorsítási szakasz
közben minnél inkább késlelteni kell a kívánt \textit{Apoapsis} elérését, így minnél inkább megemelve a \textit{Periapsis}-t.\\
Ez feltételezésem szerint akár lehet az \textit{Oberth-effektus} miatt, de ennek az igazolása nem tartozik bele a kísérlet
kereteibe.

%---------------------------------------------------------------
\subsection{A kísérletek eredménye}

Itt a \textit{MechJeb2} sokkal részletesebben bontja le a $\Delta v_{total}$-t, így az (1) egyenletben
megemlített részek is bekerülnek. Ez egy eredmény a sok közül. Több mérést is végeztem de egyik sem 
érte el a $\Delta v_{total} \leq 8900\, m/s$-ot.

\begin{table}[h]
    \centering
    \begin{tabular}{|c|c|c|c|}
        \hline
        \textbf{Paraméter} & \textbf{KSP} & \textbf{RSS} & \textbf{M.egység} \\
        \hline
        $\Delta v_{total}$ & 3176.07 & 8956.02 & m/s \\
        \hline
        $\Delta v_g$ & 529.71 & 803.89 & m/s \\
        \hline
        $\Delta v_d$ & 109.96 & 198.37 & m/s \\
        \hline
        $\Delta v_{steer}$& 264.77 & 672.12 & m/s \\
        \hline
        $\Delta v_H$ & 96.5 & 369.9 & m/s \\
        \hline
        $\gamma_f$ & 1.73 & 0 & fok \\
        \hline
    \end{tabular}
    \caption{Fontosabb mért adatok eredménye.}
    \label{tab:kspres}
\end{table}

Itt a mérés módja miatt ábrák is készíthetők. Jó sok érdekes ábrát sikerült készítsek ami érdekes 
lehet, de az összesen 14 kép lenne, így ide csak a $\Delta v$ szempontjából fontosakat rakom be.
Ha pedig valakit érdekel a többi is, \cite{OrbitProject}-ben megtalálja. Ha pedig egyéb más
összehasonlítást szeretne végezni akkor szintén fel lesznek töltve a csv fájlok amiket a
\textit{MechJeb2} készített.

\footnotetext[10]{Akit érdekel minden mért adat az a \cite{OrbitProject}-ben megtalálja.}

\begin{figure}[H]
    \centering
    \includegraphics[width=1 \textwidth]{KSP results/all\_dV(KSP).png}
    \caption{Minden $\Delta v$ fajta.}
    \label{fig:ksp_dV}
\end{figure}

\begin{figure}[H]
    \centering
    \includegraphics[width=1 \textwidth]{KSP results/all_dV_loss(KSP).png}
    \caption{Minden elvesztett $\Delta v$ fajta.}
    \label{fig:ksp_dV_loss}
\end{figure}

Már a fenti \ref{fig:ksp_dV} és \ref{fig:ksp_dV_loss} KSP ábrákból is látszik, hogy a $\Delta v_d$, 
a légellenállás $100\, m/s$-os nagyságával igazán elhanyagolható, a teljes $\approx 3200\, m/s$-hoz 
képest. És a legnagyobb a $\Delta v_g$, mint amire számítottunk. De már itt is felüti a fejét a
$\Delta v_{steer}$, ami abból származik, hogyha nem pontosan \textit{Prograde} irányba égetünk 
hanem valamilyen szögben tőle. És az elvesztett $\Delta v$ onnan származik, hogy a 
sebességvektorunkat össze kell hoznunk a haladási iránnyal. Megfigyelhető a \ref{fig:rss_dV_loss} 
ábrán is, hogy ez a $\Delta v_{steer}$ a mozgás elején jelenik meg és egy ideig több mint a $\Delta v_g$,
amíg a $\Delta v_{steer}$ konstanssá nem válik és végül a $\Delta v_g$ nagyobb lesz nála.

\begin{figure}[H]
    \centering
    \includegraphics[width=1 \textwidth]{RSS results/all_dV(RSS).png}
    \caption{Minden $\Delta v$ fajta.}
    \label{fig:rss_dV}
\end{figure}

\Aref{fig:rss_dV} ábrán megfigyelhető, hogy az arányok megváltoztak, míg \aref{fig:ksp_dV} ábrán
$\Delta v_{total} \approx 3200\, m/s$, addig egyik $\Delta v$ veszteség sem jut túlságosan túl
 az $500\, m/s$-on. Viszont \aref{fig:rss_dV} ábrán a $\Delta v_{total}$ megháromszorozódott, de a 
$\Delta v$ veszteségek nagyjából megkétszereződtek. Ez még jobban látszik \aref{fig:ksp_dV_loss} és
\aref{fig:rss_dV_loss} ábrák összevetésével.

\begin{figure}[H]
    \centering
    \includegraphics[width=1 \textwidth]{RSS results/all_dV_loss(RSS).png}
    \caption{Minden elvesztett $\Delta v$ fajta.}
    \label{fig:rss_dV_loss}
\end{figure}

A nagyjából megkétszereződés alól kivételt képez a $\Delta v_{steer}$, ami majdnem a fő égetési 
szakasz egészén nagyobb, mint a $\Delta v_g$.\\

\subsection{Összevetés}

Az én modellemben a légellenállást elhanyagoltam, mert nehéz kiszámolni és nem akkora mértékű,
sokkal fontosabb a $\Delta v_g$ kiszámolni \cite{Teofilatto}, \cite{eaglepubs}. Ez egy jó döntésnek bizonyult.\\
\indent De miért is lesz ilyen kis mértékű? Mert minél előbb el akarjuk hagyni a légkört.
Természetesen ha olyan rakétákkal számolnánk amik a Földhöz közel mozognak nagy sebességgel akkor
 nem lehetne elhanyagolni. Még felvetülhet a kérdés: De ha minél gyorsabbak vagyunk, akkor nem vesztünk
több $\Delta v$-t a légellenállás miatt? Ami igaz is, de ha lassan mennénk, akkor a gravitáció miatt
vesztenénk többet, mert a $\Delta v_g$ függ az időtől \cite{eaglepubs}. És mint azt a saját adataink 
\aref{fig:ksp_dV_loss} és \aref{fig:rss_dV_loss} ábrán mutatják a $\Delta v_g$ mindig nagyobb lesz mint
a $\Delta v_d$. Illetve ha akarnánk sem tudnánk a legsűrűbb rétegeken túl gyorsan végig menni mert a 
rakéták 1.2-1.8 \textit{TWR}-el kezdenek, ami temészetesen majd növekszik, de a tengerszint közeli
szakaszon szépen "végigsétál" a rakéta mielőtt elég sebességet gyűjetene ahhoz hogy a légellenállás 
nagyobb szerepet kapna.\\
\indent Viszont amivel nem számoltam, hogy ekkora mértékű lesz az a $\Delta v_{steer}$.
Tehát beszéljünk ennek a lehetséges okairól:\\ 
\indent Mind az \cite{eaglepubs} és \cite{Teofilatto} forrás említést tesz róla de nem állítja hogy ekkora
mértékű lenne. Én a kísérlet közben egy már az \textit{Elméleti háttér} első bekezdésében leírt módszerhez
hasonlóan végeztem el a pályára állást:\footnotemark[11]
\begin{enumerate}
    \item Kis emelkedés
    \item Bedöntés
    \item Gravitációs forduló
    \item Körpálya
\end{enumerate} 

\footnotetext[11]{A \cite{OrbitProject}-ben fel vannak videó is egy ilyen mérésről.}

Én nem alkalmaztam külön algoritmust, hanem \textit{MechJeb2}-nek egy félautomata irányítását ahol
én tudom állítani a dőlés mértékét, így követve egy közel optimális pályát.\\
Így lehet, hogy a kezdeti bedőlése a \textit{MechJeb2}-nek túl erős. Viszont ha megfigyeljük az 
eredményeket nem veszünk észre akkora különbséget. Ez lehet azért mert \citeauthor{Teofilatto}
belekalkulálták a $\Delta v_{steer}$ a számításaikba, vagy inkább éppen jól hibás a számítás, 
hogy ezt ellensúlyozza, de bizosra nem tudom megállapítani.
Természetesen mivel ez csak egy szimuláció előfordulhat, hogy nem pontos, vagy éppen a \textit{MechJeb2}
számolja rosszul a $\Delta v_{steer}$-et.

%-------------------------------------
\subsubsection*{A Bolygó forgása}

Sokáig tartogattam a $\Delta v_{lat}$-öt, hogy most kedvünkre értékelhesük ki annak a hatását.\\
A logika mögötte, hogy a Föld forog a talpunk alatt tehát van egy bizonyos kezdő kerület sebességünk
amennyit a Föld "segít" nekünk a pályára állásban\footnotemark[12].\\
Ahhoz hogy ezt vizualizálni tudjuk segít a következő \ref{fig:rss_speed} ábra. Figyeljül meg, hogy az
orbitális sebesség mindig nagyobb és a felszíni sebeség a nulla. Ez a különbség amit a Föld segít.

\begin{figure}[H]
    \centering
    \includegraphics[width=1 \textwidth]{RSS results/speed(RSS).png}
    \caption{Földi és Orbitális sebességek összehasonlítva.}
    \label{fig:rss_speed}
\end{figure}

A kiszámolása:
\begin{equation}
    \Delta v_{lat} = v_{lat} = \omega R \cos \vartheta = \frac{2\pi R}{T} \cos \vartheta
\end{equation}
Ahol $R$ a bolygó sugara, $T$ egy nap a bolygón, és $\cos \vartheta$ pedig a kilövőállás
magasságszöge\footnotemark[13]. Mivel ez mástól nem függ ezért egy konstans így
később is le lehet vonni.\\
Mivel KSP-ben az egyenlítőről indítottunk, RSS-ben pedig a Francia Guyana-ból indított adatokat
értékeltük, ezért a $\cos \vartheta$ tagot szimplán elhanyagoljuk.
\begin{table}[H]
    \centering
    \begin{tabular}{|c|c|c|c|}
        \hline
        \textbf{Pamaméter} & \textbf{KSP} & \textbf{RSS} & \textbf{M.egység} \\
        \hline
        $\Delta v_{lat}$ & 174.53 & 463.31 & m/s \\
        \hline
    \end{tabular}
    \caption{A bolygók forgásából származó sebesség.}
    \label{tab:v_lat}
\end{table} 

\footnotetext[12]{Természetesen akkor ha a forgással megegyező irányba indulunk. De hát ki akarná még
azt is leküzdeni amikor pont segítene. Ezen logikát követve a keletre indítás a konvenció.}

\footnotetext[13]{Vagy Alt-szög, szélesség, vagy latitude}

%-------------------------------------
\subsubsection*{Adatok összehasonlítása}

Ha az előbb kiszámolt eredményket egymás mellé rakjuk a következő táblázatot kapjuk:

\begin{table}[H]
    \centering
    \begin{tabular}{|c|c|c|c|}
        \hline
        \textbf{Eredménye} & \textbf{KSP} & \textbf{RSS} & \textbf{M.egység} \\
        \hline
        Kísérlet & 3176.07 & 8956.02 & m/s \\
        \hline
        Számítás & 3223.72 & 8930.49 & m/s \\
        $ - \Delta v_{lat}$ & 3049.19 & 8467.18 & m/s \\
        \hline
    \end{tabular}
    \caption{$\Delta v_{total}$ a bolygók forgását beszámítva és az eredeti eredményekkel öszehasonlítva.}
    \label{tab:res_comp}
\end{table}

Így a számítások azt sugallják, hogy ezek a $\Delta v_{total}$ értékek elegendőek lennének a pályára álláshoz.
De ezt kísérlettel nem tudom alátámasztani. Nem kizárt hogy KSP-ben lehetne még $\approx 150\, m/s$-ot faragni,
de ez a Föld közel $\approx 500\, m/s$-ra már nem mondható el. De nem is feltétlen az a kérdés, hogy fizikailag
lehetséges-e, hanem az hogy érdemes lenne egy ennyi $\Delta v_{total}$-vel rendelkező rakétát tervezni. Amire a 
válasz, hogy nem, hiszen ezek a minimum értékek. Tehát \textit{lehet}, de a biztonság és egyéb ki nem számítható
nehézségek és hibák miatt nem érdemes cseppre kimérni az üzemanyagot.\\
Valamint a számítás, részben analitikus mivolta miatt bizonyos egyszerűsítéseket alkalmaz, tehát lehet azt 
mondani, hogy a kimért adatok alapján pont annyi a hiba amennyi a $\Delta v_{lat}$ lenne.\\
De amennyiben a játékkal játszunk, még érdemes ennek a tetejébe is egy kis plusz üzemanyagot vinni 
$\approx 200-400\, m/s$-ot. Ezeket a $\Delta v$ térképek adatai alapján mondom, \cite{KSP_deltaV_map}, 
\cite{RSS_deltaV_map} amiket a Kerbal Space Program közösség készített az évek alatt.

%-------------------------------------
\subsubsection*{Alternatív ötlet}
Egyszer a kód hibás volt és úgy viselkedett ami a Hohmann módszerre emlékeztetett engem. Így jutott eszembe
ez a megoldás.
\medskip

\noindent Mi lenne, ha Hohmann módszerrel számolnánk $\Delta v_{total}$-t?\\
Semmi mást nem kell tennünk, mint kiszámoljuk, mennyi a kerületi sebesség $R$ magasságban, majd a 
Hohmann módszert alkalmazva felmegyünk egy $R+h$ magasságú kör pályára.
\smallskip

Ez lenne a leghatékonyabb módja a pályára állásnak, ha nem lenne mindenféle veszteségünk, azért mert ki kell
kerüljük a légkört, illetve fel kell emelkedjünk. Mert nem egy pillanat alatt fogunk $7.9\, km/s$-ra szert tenni.
\smallskip
A Python kód ezt is kiszámolja, így adódik a következő táblázat:

\begin{table}[H]
    \centering
    \begin{tabular}{|c|c|c|c|}
        \hline
        \textbf{Eredménye} & \textbf{KSP} & \textbf{RSS} & \textbf{M.egység} \\
        \hline
        Hohmann & 2735.33 & 8134.84 & m/s \\
        \hline
        Hohmann + $\Delta v_g$ & 3386.82 & 9237.58 & m/s \\
        \hline
        Hohmann + $\Delta v_g$ - $\Delta v_{lat}$ & 3212.29 & 8774.27 & m/s \\
        \hline
    \end{tabular}
    \caption{Tisztán Hohmann módszerrel a $\Delta v_{total}$, és $\Delta v_g$ beleszámolva.}
    \label{tab:hohmann}
\end{table}

Az eredmény bíztató, mert nem kaptunk tisztán Hohmann módszer alatti értéket a számításaink-ban, ami azt jelentette volna, hogy
hibás a számításunk. A másik két sor pedig alkalmazza az (1) gondolatmenetét, miszerint, hogyha valahogyan
tudnánk $\Delta v_g$-t akkor a $\Delta v$ fajták összeadásával megkapjuk $\Delta v_{total}$-t. Így megint 
csak egészen közeli értékeket kapunk a valósághoz.\footnote[14]{A $\Delta v_g$ a Python kód által megállapított érték.}
\medskip

Ha még pontosabb eredményeket akarunk kapni arra egy teljesen numerikus megközelítés, ami nem hanyagolja el a légellenállást
és a $\Delta v_{steer}$-et megoldást nyújthat.

%-------------------------------------------------------------------------------------------
\section{Összefoglalás}

A projekt célja az volt, hogy állapítsuk meg az alacsony Föld és Kerbin körüli pályára álláshoz szükséges 
$\Delta v$-t. \citeauthor{Teofilatto} munkája alapján a pályára állás egy részét analitikusan megoldva, majd a 
kezdőfeltételeket numerikusan hozzáigazítva a kívánt pályaadatokhoz sikerült megállapítani a Föld és
Kerbin körüli pályára állás $\Delta v$ szükségleteit. Ezeket a körülbelüli eredményeket pedig 
Kerbal Space Programmal sikerült alátámasztani, és visszaigazolni a közösség által ajánlott értékeket.

%-------------------------------------------------------------------------------------------
\nocite{*}
\printbibliography

\end{document}